%****** Wrong sign Lagrangian on the  Worldsheet ******
\documentclass[%
 reprint,
 amsmath,amssymb,
 aps,
]{revtex4-1}
\usepackage{graphicx}% Include figure files
\usepackage{dcolumn}% Align table columns on decimal point
\usepackage{bm}% bold math
\usepackage{hyperref}
%\newcommand{\be}{\begin{equation}}
%\newcommand{\ee}{\end{equation}}
%\newcommand{\bea}{\begin{eqnarray}}
%\newcommand{\eea}{\end{eqnarray}}
%\newcommand{\bg}{\begin{gather}}
%\newcommand{\eg}{\end{gather}}
%\newcommand{\bseq}{\begin{subequations}}
%\newcommand{\eseq}{\end{subequations}}
%\newcommand{\tg}{\mathop{\rm tg}\nolimits}
%\renewcommand{\tanh}{\mathop{\rm th}\nolimits}
%\newcommand{\ch}{\mathop{\rm ch}\nolimits}
%\renewcommand{\ln}{\mathop{\rm ln}\nolimits}
%\newcommand{\sm}[1]{{\scriptscriptstyle \rm #1}}
%\newcommand{\Tr}{{\rm Tr}}
%\def\half{\frac{1}{2}}
%\newcommand{\eq}[1]{(\ref{#1})}
%\newcommand{\goe}{\gtrsim}
%\newcommand{\loe}{\lissome}
%\newcommand{\bra}[1]{\langle #1 |}
%\newcommand{\ket}[1]{| #1 \rangle}
%\newcommand{\p}{\partial_+}
%\newcommand{\m}{\partial_-}
%\newcommand{\newj}{\bar{\updelta}_0^j}
%\newcommand{\pd}{\partial}
\begin{document}

\title{Acausality on the String Worldsheet}

\author{Patrick Cooper}
 \email{pjc370@nyu.edu}
 \affiliation{Center for Cosmology and Particle Physics, Department of Physics, New York University, New York, NY, 10003, USA}
\author{Sergei Dubovsky}
 \email{sergei.dubovsky@gmail.com}
 \affiliation{Center for Cosmology and Particle Physics, Department of Physics, New York University, New York, NY, 10003, USA}
\author{Ali Mohsen}
 \email{ahm302@nyu.edu}
\affiliation{Center for Cosmology and Particle Physics, Department of Physics, New York University, New York,
NY, 10003, USA}

\date{\today}

\begin{abstract}
    In this paper we explore a peculiar theory with an acausal, but otherwise healthy S-Matrix and a
    perfectly well defined local Lagrangian.  This S-matrix corresponds to the integrable theory of
    worldsheet excitation scattering on a long string in the static gauge, however with a negative
    string tension (or imaginary width). We explore how this acausality rears its  head in many
    features of the classical and quantum theory and use the Thermodynamic Bethe Ansatz (TBA) to
    probe thermodynamic behavior.
\end{abstract}

\maketitle

%\tableofcontents

\section{\label{sec:level1}Introduction}
While investigating the worldsheet dynamics of a long closed string, it was found in
\cite{Dubovsky:2012wk} that left and right moving excitations of the worldsheet, whose dynamics are
governed by the Nambu-Goto action, are described by an integrable reflectionless 2D QFT with the following
S-matrix, diagonalized in the momentum basis \begin{eqnarray} e^{2 i \delta(s)} = e^{i s l_s^2/4} ,
\nonumber \end{eqnarray} where $s$ is the usual Mandelstam variable, and $l_s \equiv 1/2 \pi T$ is
the string scale. It was shown that this theory shares many of the coveted features of quantum
gravity, including a prototype for black hole formation and evaporation, as well as an inability to
construct local observables. As one can check, the time delay of scattering in the center of mass
frame of 2 wave packets of Nambu-Goldstone bosons on the worldsheet is proportional to $\frac{1}{2}
E l_s^2$.

At this point, an interesting question arises. From the perspective of a theory of D scalar fields on the
worldsheet, the string tension is simply a coupling constant.  What is the effect of reversing the
sign of this interaction?  To do this consistently, we go to the static gauge, and canonically
normalize the kinetic term leading to the following Lagrangian
\begin{eqnarray}
\label{action}
\mathcal{L} = -T \sqrt{ - \mathrm{det} (\eta_{a b} - \frac{1}{T} \partial_a X^{\mu}
    \partial_b X^{\nu} \eta_{\mu \nu})} \nonumber \\
 = -T \left( 1 - \frac{1}{T}
     (-(\dot{X})^2 + (X')^2) + \right. \nonumber \\
     \left. \frac{1}{T^2} ((\dot{X} \cdot X')^2 -
     (\dot{X})^2 (X')^2) \right)^{1/2} .
\end{eqnarray}
This Lagrangian corresponds by analogy to the following S-Matrix
\begin{eqnarray}
    e^{-i s l_s^2 /4} . \nonumber
\end{eqnarray}
In this theory we expect time advanced scattering as opposed to time delayed, ruining the causality
of the theory.  This may seem surprising since, on the face of it, the Lagrangian can be expanded in a
Taylor series, making manifest the fact that all interactions are perfectly local and Lorentz invariant.  It was shown
in \cite{Adams:2006sv} however, that there are additional constraints on an effective Lagrangian.
Namely for a scalar field theory with a shift symmetry, the leading quartic term must have postive
coupling. What's interesting with this particular example, is that we have more than an effective
Lagrangian.  In fact, we know that the target space can be viewed as a dynamical embedded surface in
a D-dimensional spacetime and thus it can be quantized and subsequent time evolution can be studied.
With these tools, as was shown in \cite{Dubovsky:2012wk}, one is able to extract an exact S-matrix
for the \emph{embedded} worldsheet theory. Even though the theory with the flipped coupling constant is an
effective theory from the perspective of perturbation theory, by deriving the full S-matrix we have
in a sense UV completed this theory. Thus it is an ideal playground to isolate the effects of
non-positivity \cite{Adams:2006sv}.  We do so by redoing some of the calculations found in
\cite{Dubovsky:2012wk} and \cite{Adams:2006sv} with our new theory and extract physical intuition
about where pathologies become manifest.

%*******************************
\section{Classical Scattering}
%*******************************

This Lagrangian (\ref{action}), has a perfectly well defined, canonically normalized derivative expansion in the fields
$\partial_{a}X^i$ of the form
\begin{eqnarray}
    \mathcal{L} = \frac{1}{2} \partial_{a} X \cdot \partial^{a} X - \frac{1}{8 T}
    (\partial_{a} X \cdot \partial^{a} X)^2 + \nonumber \\
    \frac{1}{2 T^2} ((\dot{X} \cdot X')^2 - (\dot{X}^2 \cdot X'^2)) + \mathcal{O}(X^8) , \nonumber
\end{eqnarray}
where the dot represents the usual Euclidean inner product on the flavor space of the transverse degrees of freedom and
contraction of Latin worldsheet indices is done via the flat Minkowski metric with signature $(1,1)$. Note we subtracted
the cosmological constant $-T \cdot \mathrm{Area}$ which is infinite and non-dynamical in the free 2D theory of a long
effective string. As was mentioned in \cite{Adams:2006sv}, effective field theories with a shift invariance and negative
leading quadratic coupling may appear to be well behaved in the sense that they have a Lorentz invariant vacuum, however
when scattering excitations on non-trivial backgrounds, superluminal signal propagation becomes manifest in the
excitation's dispersion relations.

For simplicity we'll provide an example for a string in codimension one, where the Lagrangian
density for the single transverse mode parameterizing perturbations of the string from equilibrium, can be written
\begin{eqnarray}
    \mathcal{L} = \frac{1}{2} (\partial_{a} X)(\partial^{a} X) -
    \frac{1}{8 T}[(\partial_{a} X)(\partial^{a} X)]^2 + \mathcal{O}(X^8) \nonumber .
\end{eqnarray}
The field equations resulting from minimizing the action take the following form
\begin{eqnarray}
    \square X \left( 1- \frac{1}{2 T} (\partial_a X) (\partial^a X) \right) - \frac{1}{T} (\partial_a \partial_b X)
    (\partial^a X \partial^b X) \nonumber \\
    = 0. \nonumber
\end{eqnarray}
Now let us see how how when scattering off of non-trivial backgrounds, pathologies emerge.  Consider a large classical background,
$\bar{X}$, such that $\partial_a \bar{X} = C_a$ for some constant vector $C$.  The linearized
equations of motion of the field $\phi$ defined by $X = \bar{X} + \phi$ take the form
\begin{eqnarray}
    \square \phi \left( 1 - \frac{1}{2 T} C_a C^a \right) - \frac{1}{T} (\partial_a \partial_b \phi)
    (C^a C^b) = 0. \nonumber
\end{eqnarray}
Expanding $\phi$ into plane-waves, this equations leads to the following expression in Fourier space
\begin{eqnarray}
    -p^2 \left( 1 - \frac{1}{2 T} C_a C^a \right) + \frac{1}{T} (k \cdot C)^2 = 0 \nonumber \\
        \rightarrow -p^2 = - \frac{(k \cdot C)^2}{T - \frac{1}{2} C_a C^a} \; .  \nonumber
\end{eqnarray}
These dispersion relations are superluminal for $T > \frac{1}{2} C_a C^a$.

To see how these interactions with a non-trivial background manifest themselves semi-classically in
a time advance, consider a large left moving solution to the NG equations of motion. This implies
that $\bar{X}$ is a function of $\sigma_+ = \sigma + \tau$ only and $\partial_{\sigma} \bar{X} =
\partial_{\tau} \bar{X} = \bar{X}'$ where $\bar{X}'$ is simply the derivative of X with respect to its argument.
Again, we consider the codimension one case to avoid carrying flavor indices.

Scattering of a small right mover off of this background amounts to studying the null geodesics
on the worldsheet with an induced metric given by our classical solution (now omitting bars over
X and setting $T = 1$ for brevity)
\begin{eqnarray}
    \mathrm{ds}^2 &=& \left( \eta_{a b} - \frac{1}{T} \partial_a X \partial_b X \right)
    \mathrm{d\sigma}^a \mathrm{d\sigma}^b \nonumber \\ &=& (-1-X'^2)\mathrm{d \tau}^2 - 2 X'^2 \mathrm{d \sigma}
    \mathrm{d \tau} + (1 - X'^2) \mathrm{d \sigma}^2 \nonumber .
\end{eqnarray}
The null geodesic equation is given by
\begin{eqnarray}
    \dot{\sigma}(\tau) = \frac{X'^2 \pm 1}{-X'^2 + 1} \nonumber ,
\end{eqnarray}
and we chose the upper sign for a right moving excitation which will experience a non-trivial time
shift. Already we can see this is a time advance since $\dot{\sigma} > 1$.  This advance can be
calculated by considering the following
\begin{eqnarray}
    \Delta t &=& \int_{\infty}^{\infty} d\tau (1 - \dot{\sigma}) = \int_{\infty}^{\infty} d\tau \frac{-2
    X'^2}{-X'^2 + 1} \nonumber \\
    &=& - \int_{\infty}^{\infty} d\sigma_+ X'^2 (\sigma_+) \sim -l_s^2 \Delta E , \nonumber
\end{eqnarray}
where in the last line we incorporated the string scale to restore units and $\Delta E$ is the
energy of the classical solution relative to the vacuum. We see that the time shift between the
scattered solution and the freely propagating solution is negative, implying a classical
superluminal effect.

%**************************
\section{Thermodynamics}
%**************************

So far, we have extracted dispersion relations with the Lagrangian and verified time advances in the
classical theory suggested by the form of the S-Matrix. Another hallmark of any theory is the energy
spectrum. To obtain this from the perspective of a 2 dimensional field on the
worldsheet, we use the Thermodynamic Bethe Ansatz (TBA), $\mathrm{\grave{a}}$ la \cite{Dubovsky:2012wk}.  TBA
exploits the symmetric topology of $S^1 \times S^1$. Consider a (1+1) dimensional field theory spanned in space by the first
$S^1$ with radius R, and take the radius L of the other $S^1$ (the timelike circle) to infinity. The partition function of
this theory is then dominated by the ground state energy
\begin{eqnarray}
    Z(R,L) \sim e^{-L E(R)} \nonumber .
\end{eqnarray}
Similarly, if we were to compactify on the circle of radius L, the $L \rightarrow \infty$ limit
would correspond to the thermodynamic limit of a theory with finite temperature, $1/R$ .  The
partition function of this theory is
\begin{eqnarray}
    Z(R,L) = \mathrm{Tr} \left( e^{-R H_L} \right) \nonumber ,
\end{eqnarray}
where $H_L$ is given by the Hamiltonian of this representation. By equating the two partition
functions, as shown in \cite{Zamolodchikov:1989cf}
\begin{eqnarray}
    E(R) = R f(R) \nonumber ,
\end{eqnarray}
where $f(R)$ is the free energy of the system.  Now the task of finding the ground state energy of
our system of bosons is equivalent to the task of finding the free energy. For this we again follow
\cite{Zamolodchikov:1989cf} and \cite{Dubovsky:2012wk} and consider the wave function as consisting
of several free wave functions, with the S-Matrix providing us with the appropriate matching conditions patching
them together.  The resulting integral equation constraining the particle densities is then given by
\begin{eqnarray}
    2 \pi \rho_{R,L}^i(p) = L - \l_s^2 \sum_{j=1}^{D-2} \int_0^{\infty} p' \rho_{1L,R}^j (p') dp' ,
\end{eqnarray}
where $\rho_{1R,L}$ is the density of right and left moving particles respectively and $\rho_{L,R}$ is
the level density. To calculate the free energy subject to the constraints above, we have to
find the saddle point of the partition function defined as
\begin{eqnarray}
    Z(R,L) = \int \prod_i \mathcal{D} \rho^i_L \mathcal{D} \rho^i_R \; \mathrm{exp}(- R H[\rho^i_{1L},
    \rho^i_{1R}] + \nonumber \\
    S[\rho^i_{1L}, \rho^i_{1R},\rho_L,\rho_R] ) \nonumber .
\end{eqnarray}
The calculation follows exactly along the lines of \cite{Dubovsky:2012wk} only with negative vacuum energy
on the worldsheet and slightly different equations for the pseudoenergies. The free energy is then given by
\begin{eqnarray}
    F = - \frac{L}{l_s^2} + \frac{L}{2 \pi R} \sum_{j=1}^{D-2} \int_0^{\infty} d p'
    \mathrm{ln}\left(1 - e^{-R \epsilon_L^j (p')} \right) \nonumber \\
    + \frac{L}{2 \pi R} \sum_{j=1}^{D-2} \int_0^{\infty} d p'
    \mathrm{ln}\left(1 - e^{-R \epsilon_R^j (p')} \right) \nonumber ,
\end{eqnarray}
where the functions $\epsilon_{L,R}$ are given by
\begin{eqnarray}
    \epsilon_L^i (p) = p \left[ 1 - \frac{l_s^2}{2 \pi R} \sum_{j=1}^{D-2} \int_0^{\infty}
    d p' \mathrm{ln} \left(1 - e^{-R \epsilon_R^j (p')}\right) \right] \nonumber \\
    \epsilon_R^i (p) = p \left[ 1 - \frac{l_s^2}{2 \pi R} \sum_{j=1}^{D-2} \int_0^{\infty}
    d p' \mathrm{ln} \left(1 - e^{-R \epsilon_L^j (p')}\right) \right] \nonumber .
\end{eqnarray}
The ground state energy is then given by the solution to this series of equations which is connected
to the free theory.  The result is
\begin{eqnarray}
    E_0(R) = \frac{R}{L} F(R) = -\sqrt{\frac{R^2}{l_s^4} + \frac{4 \pi}{l_s^2} \frac{D-2}{12}}
    \nonumber .
\end{eqnarray}
As expected, this expression closely resembles the ground state energy of Bosonic string theory
found in the light cone gauge. One major difference however, is that the free energy is real at
every temperature ($1/R$).  This means that the Hagedorn behavior of canonical worldsheet dynamics
isn't present in this theory.  In a sense, the thermodynamics of this superluminal theory is less
pathological.

Hagedorn behavior arises in string theory because the exponential degeneracy of states becomes
closer and closer in energy with higher and higher string excitations. This leads to a divergence in
the density of states, which in turn can be seen as a divergence of the heat capacity in the
thermodynamic limit of the theory.  Here we have no such divergence in the second derivative of the
free energy.  This is because additional excitations of the negative tension string seem to 'repel'
each other in the sense that there is negative binding energy. To see this we need to derive the
full energy spectrum of the theory.

If one repeats the previous exercise for excited states, which amounts to modifying the TBA
constraint equations by a term resembling the addition of a chemical potential, as in
\cite{Dubovsky:2012wk}, one obtains the following energy spectrum
\begin{eqnarray}
    \label{enspec}
    E(N,\tilde{N}) = -\left( \frac{4 \pi^2 (N - \tilde{N})^2}{R^2} + \frac{R^2}{l_s^4} \right .
    \nonumber \\ \left . -\frac{4 \pi}{l_s^2} \left( N + \tilde{N} - \frac{D-2}{12} \right)
    \right)^{1/2} .
\end{eqnarray}
The lack of Hagedorn behavior in this wrong sign theory is now clear. Examining (\ref{enspec}) in
the limit of $R\to\infty$, we conclude that the interaction can be interpreted as repulsive. For
instance if we take $\Delta N >> 1$ right movers and left movers, we find that
\begin{eqnarray}
   E_{\Delta N,0}+E_{0,\Delta N}-2E_{0,0} = -\frac{4\pi \Delta N}{R}  \nonumber \\
   E_{\Delta N,\Delta N} - E_{0,0} \approx +\frac{4\pi \Delta N}{R} \nonumber
\end{eqnarray}
Thus we have negative binding energy $ (E_{\Delta N,0}+E_{0,\Delta N}-2E_{0,0})-(E_{\Delta N,\Delta
N} - E_{0,0} )<0 $. This is to be contrasted with the $\mathcal{O}(1/R^2)$ binding energy found in
the $+l_{s}^{2}$ theory.

Another striking feature of this theory is that the spectrum resists compactification. To see this,
we rewrite the expression beneath the square root as
\begin{eqnarray}
    E^2 \propto \left( N - \tilde{N} \right)^2 + \frac{1}{4} a^2 - a \left( N + \tilde{N} - b \right)
    \nonumber ,
\end{eqnarray}
where $a = \left( R / l_s \right)^2 1 / \pi $ is the size of the compactified dimension relative to
the string scale and $ b = (D-2)/12 $ is related to the number of different flavors. We see that at
fixed $a>1$, the Hamiltonian becomes non-hermitian as we start increasing the number excitations
above the vacuum, ie $E^2 < 0$. This appears to be the quantum analogue of closed time-like curves
that can be constructed with classical scattering. If we attempted to compactify the classical
theory on a circle and chose the size of the circle such that $R = \Delta t$, we would create a
closed timelike curve.  To get a relation between the classical energy of the kink and the number of
movers, we assume in the semiclassical limit $E$ is dominated by the Kaluza-Klein energy, i.e. $E =
2 \pi N / R $.  This is justified by the large N behavior of the energy spectrum.  Solving this
condition for closed
timelike curves for the critical radius as a function of the size ($N$) of the kink, we arrive at
\begin{eqnarray}
    R = l_s^2 \frac{2 \pi N}{R} \rightarrow R_{crit} = \sqrt{2 \pi N} l_s \; . \nonumber
\end{eqnarray}
Likewise, in the quantum theory, we could calculate where $E^2$ becomes negative in the limit $N >>
\tilde{N} >> 1$ and we find precisely that $R_{crit} = \sqrt{2 \pi N} l_s$.

An interesting remark is that the condition that $E^2$ vanish at some value of $N$ even for a
minimally perturbed system with $\tilde{N}$=1, is that $D \leq 26$. Thus the critical dimension
seems to play a role in truncating the spectrum to avoid closed timelike curves.

Due to the lack of Hagedorn behavior, we expect a non-singular equation of state.  From the free
energy, we see that the pressure dependence of the energy density becomes
\begin{eqnarray}
    \label{eos}
    \rho = \frac{p}{1+l_s^2 p} .
\end{eqnarray}
This is just a $\sigma \leftrightarrow \tau$ transformation of the stress energy tensor,
swapping $\rho$ and $p$ of the canonical positive Nambu-Goto theory. This is a recurring theme;
notice that the spacetime diagram of null geodesics in the classical scattering off of a kink
also undergoes this transformation when the sign of the coupling reverses.  For strings, this
intuition holds because swapping the sign of $l_s^2$ is equivalent to shifting the target space
metric from $(-,+, \cdots , +)$ to $(-,+,-, \cdots,-)$, effectively reversing the role of $\tau$ and
$\sigma$. The target space interpretation becomes unclear for general $p$-branes, where the
generalization of this argument leads to multiple time dimensions.

%*********************
\section{Cosmology}
%*********************

Another calculation to shed light on the bizarre nature of this theory, in the spirit of
\cite{Dubovsky:2012wk}, is to interpret cosmological solutions on the wrong sign string worldsheet.
In (1+1) dimensions, isotropy isn't a constraint, so the only notion of "rotation invariance" we can
have is "boost invariance" or rotations in the $\sigma\tau$ plane.  Homogeneity is over
constraining, leaving us only with the trivial vacuum solution so we seek solutions that are
functions of the invariant length on the worldsheet, $X \equiv X(-\tau^2 + \sigma^2)$.  Again, to
simplify the discussion, we'll consider only one additional flavor to those that can be fixed by a
gauge transformation. The equations of motion for (\ref{action}), using hyperbolic coordinates are
\begin{equation}
    \partial_{\alpha^2} \left( \frac{\alpha^2 \partial_{\alpha^2} X}{\sqrt{1 -
    \frac{4\alpha^2}{T}(\partial X)^2}} \right) = 0 \nonumber \; ,
\end{equation}
with $\alpha^2 = -\tau^2 + \sigma^2$.  For the $+l_{s}^{2}$ theory (normalizing $l_s^2$ for brevity), the
solution consisted of two branches,
$\mathcal{A}$ and $\mathcal{B}$
\begin{align}
    \mathrm{Branch} \, \mathcal{A}: \; \; \left . X^{0}_A \right . ^{2} -\left . X^{1}_A \right . ^{2} -
    (\text{sinh } X_A)^{2}=0 \nonumber \\
    \mathrm{Branch} \, \mathcal{B}: \; \; \left . X^{1}_B \right . ^{2} -\left . X^{0}_B \right . ^{2} -
    (\text{cosh } X_B)^{2}=0 \nonumber
\end{align}
Switching the sign of the string tension doesn't change the solutions to the field equations,
however it does alter their interpretation.  Recall from the last section that if we view the Goldstone
fields as embedding coordinates in an ambient spacetime (which we're implicitly doing here by
writing the solutions as a hypersurface in the 3 dimensions spanning flavor space) reversing the
sign of the derivative coupling effectively reverses the role of space and time, \emph{ie.} the role
of $X^0$ and $X^1$. Let us examine the effects of this on the cosmological interpretation of
the solutions.

As in \cite{Dubovsky:2012wk}, we define for branches $\mathcal{A}$ and $\mathcal{B}$, $X_A^0 = \rho
\, \mathrm{cosh} \lambda$, $X_A^1 = \rho \, \mathrm{sinh} \lambda$, and $X_B^0 = \rho
\, \mathrm{sinh} \lambda$, $X_B^1 = \rho \, \mathrm{cosh} \lambda$ respectively. This leads to the
following induced metric on the worldsheet for solution $\mathcal{A}$
\begin{eqnarray}
    ds^2_A = -\frac{\rho^2}{1+\rho^2} d\rho^2 + \rho^2 d \lambda^2 = (\tau^2 + 2 \tau)
    d\lambda^2 - d \tau^2 \nonumber ,
\end{eqnarray}
with $\tau = \sqrt{1 + \rho^2} -1$. For $\mathcal{B}$,
\begin{eqnarray}
    ds^2_B =  - \rho^2 d \lambda^2 + \frac{\rho^2}{\rho^2 - 1} d \rho^2 = dr^2 - (1 + r^2) d \lambda^2
    \nonumber ,
\end{eqnarray}
with $r = \sqrt{\rho^2 - 1} $.  The interpretation of these solutions is inverted from the positive
theory.  We now should consider our metric to be mostly minuses, and thus $r$ and $\tau$ are time
and space respectively. Now branch $\mathcal{A}$ is a static solution, and branch $\mathcal{B}$ is a
cosmological solution, but this time with no big bang singularity.  The microscopic understanding of
this lack of singularity can be thought of as the remnant of the wrong sign coupling leading to
negative binding energies.  This repulsion manifests itself as a minimal size of the worldsheet as
one interpolates in time from $-\infty$ to $+\infty$.

\section{Acknowledgements}

We'd like to thank Victor Gorbenko for his valuable input during this project.

\bibliography{superlum}

\end{document}
