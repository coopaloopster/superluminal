%****** Wrong sign Lagrangian on the  Worldsheet ******
\documentclass[%
 reprint,
 amsmath,amssymb,
 aps,
]{revtex4-1}
\usepackage{graphicx}% Include figure files
\usepackage{dcolumn}% Align table columns on decimal point
\usepackage{bm}% bold math
\usepackage{hyperref}
%\newcommand{\be}{\begin{equation}}
%\newcommand{\ee}{\end{equation}}
%\newcommand{\bea}{\begin{eqnarray}}
%\newcommand{\eea}{\end{eqnarray}}
%\newcommand{\bg}{\begin{gather}}
%\newcommand{\eg}{\end{gather}}
%\newcommand{\bseq}{\begin{subequations}}
%\newcommand{\eseq}{\end{subequations}}
%\newcommand{\tg}{\mathop{\rm tg}\nolimits}
%\renewcommand{\tanh}{\mathop{\rm th}\nolimits}
%\newcommand{\ch}{\mathop{\rm ch}\nolimits}
%\renewcommand{\ln}{\mathop{\rm ln}\nolimits}
%\newcommand{\sm}[1]{{\scriptscriptstyle \rm #1}}
%\newcommand{\Tr}{{\rm Tr}}
%\def\half{\frac{1}{2}}
%\newcommand{\eq}[1]{(\ref{#1})}
%\newcommand{\goe}{\gtrsim}
%\newcommand{\loe}{\lissome}
%\newcommand{\bra}[1]{\langle #1 |}
%\newcommand{\ket}[1]{| #1 \rangle}
%\newcommand{\p}{\partial_+}
%\newcommand{\m}{\partial_-}
%\newcommand{\newj}{\bar{\updelta}_0^j}
\newcommand{\pd}{\partial}
\begin{document}

\title{Acausality on the String Worldsheet}

\author{Sergei Dubovsky}
 \email{sergei.dubovsky@gmail.com}
 \affiliation{Center for Cosmology and Particle Physics, Department of Physics, New York University, New York, NY, 10003, USA}
\author{Patrick Cooper}
 \email{pjc370@nyu.edu}
\affiliation{Center for Cosmology and Particle Physics, Department of Physics, New York University, New York, NY, 10003, USA}
\author{Ali Mosen}
 \email{ahm302@nyu.edu}
\affiliation{Center for Cosmology and Particle Physics, Department of Physics, New York University, New York,
NY, 10003, USA}

\date{\today}

\begin{abstract}
    In this paper we explore a peculiar theory with an acausal, but otherwise healthy S-Matrix and a perfectly well
    defined local Lagrangian.  This S-matrix corresponds to the integrable theory of worldsheet excitation scattering on
    a long string in the static gauge, however with a negative string tension (or imaginary width). We explore how this
    acausality rears its  head in many features of the classical and quantum theory and use the Thermodynamic Bethe
    Ansatz (TBA) to probe thermodynamic behavior.
\end{abstract}

\maketitle

%\tableofcontents

\section{\label{sec:level1}Introduction}

\begin{equation}
    \label{action}
    \mathcal{S} = T \int d^{p+1} \sigma \sqrt{-\mathrm{det}(\eta_{a b} - \frac{1}{T_0} \partial_{a} X^i \partial_{b}
    X^i)} .
    \nonumber
\end{equation}

%*******************************
\section{Classical Scattering}
%*******************************
This Lagrangian (\ref{action}), has a perfectly well defined, canonically normalized derivative expansion in the fields
$\partial_{a}X^i$ of the form
\begin{eqnarray}
    \mathcal{S} = \int d^{p+1} \sigma \left( -\frac{1}{2} \partial_{a} X \cdot \partial^{a} X - \frac{1}{8 T}
    (\partial_{a} X \cdot \partial^{a} X)^2 + \right . \nonumber\\
    \left . \frac{1}{2 T} ((\dot{X} \cdot X')^2 - (\dot{X}^2 \cdot X'^2)) + \mathcal{O}(X^8) \right) ,
    \nonumber
\end{eqnarray}
where the dot represents the usual Euclidean inner product on the flavor space of the transverse degrees of freedom and
contraction of Latin worldsheet indices is done via the flat Minkowski metric with signature $(1,p)$. Note we subtracted
the cosmological constant $T \cdot \mathrm{Area}$ which is infinite and non-dynamical in the free 2D theory of a long
effective string. As was mentioned in \cite{Adams:2006sv}, effective field theories with a shift invariance and negative
leading quadratic coupling may appear to be well behaved in the sense that they have a Lorentz invariant vacuum, however
when scattering excitations on non-trivial backgrounds, superluminal signal propagation becomes manifest in the
excitation's dispersion relations.

We'll consider a string of length a in codimension one, where the Lagrangian for the single transverse mode
parameterizing the perturbation of the string from equilibrium can be written,
\begin{equation}
    L[y(\sigma,\tau)] = -T \int_0^a d\sigma \sqrt{1 + y'^2(\sigma,\tau) - \dot{y}^2(\sigma,\tau)} \nonumber.
\end{equation}
where prime and dot denote the usual derivative with respect to $\sigma$ and $\tau$ respectively. After reversing the
sign of the interaction, canonically normalizing the kinetic term, and expanding the square root, the non-constant
Lagrangian becomes
\begin{equation}
    L \rightarrow \int_0^a d\sigma (-\frac{1}{2} (\partial_{a} y)(\partial^{a} y) -
    \frac{1}{8 T}[(\partial_{a} y)(\partial^{a} y)]^2 + \mathcal{O}(y^8)) \nonumber.
\end{equation}
Making the substitution $T = 1/(2 \pi l_s^2)$, the variation of this action leads to the following equations of motion,
\begin{equation}
    \partial^2 y + \pi l_s^2 ((\partial y)^2 \partial^2 y + 2 \pi l_s^2 (\partial \partial y) \cdot (\partial y \partial
    y)) = 0 \nonumber .
\end{equation}
Consider now scattering of an excitation of $y$ off of a constant background, $\overline{y}$.  Define $\partial_{a}
\overline{y} = C_{a}$ and $y = \phi + \overline{y}$.
The linearized equations of motion take the form
\begin{equation}
    \partial^2 \phi + 6 \pi l_s^2 ( C_{a} C^{a} \partial^2 \phi + 2 (\partial_{a} \partial_{b} \phi) C^{a}
    C^{b}) = 0 .
\end{equation}
In momentum space, this leads to the following superluminal dispersion relation and hence, speed of sound
\begin{equation}
    c_s^2 \equiv \frac{\omega^2}{k^2} = 1 + 12 \pi l_s^2 (k \cdot C)^2 \nonumber .
\end{equation}
This is to be expected from a similar calculation done in \cite{Adams:2006sv}, where it was shown that the sign of the
leading interaction can often lead to macroscopic non-locality, despite the fact that elementary perturbations of the
vacuum state exhibit no such pathological behavior.  We

%This behavior can be seen in a classical spacetime diagram of a Goldstone boson with a kink soliton solution.

% Add here the explanation and graph of scattering across a kink solution. Include metric calculation and all that jazz

%*********************
\section{Cosomology}
%*********************

Another calculation to shed light on the bizarre nature of this theory, in the spirit of \cite{Dubovsky:2012wk},
is to interpret cosmological solutions on the wrong sign string worldsheet. In (1+1) dimensions, isotropy isn't
a constraint, so the only notion of "rotation invariance" we can have is "boost invariance" or rotations in the
$\sigma-\tau$ plane.  Homogeneity is over constraining, leaving us only with the trivial vacuum solution so we seek
solutions that are functions of the invariant length on the worldsheet, $X \equiv X(-\tau^2 + \sigma^2)$.  Also, to
simplify the discussion, we'll consider only one additional flavor to those that can be fixed by a gauge
transformation. The equations of motion for \ref{action}, using hyperbolic coordinates
\begin{equation}
    \partial_{\rho^2} \left( \frac{\frac{1}{T_0} \partial_{\rho^2} X \rho^2}{\sqrt{-1 \pm \frac{1}{T_0} \rho^2 (\partial X)^2}} \right) = 0
\end{equation}
with $\rho^2 = -\tau^2 + \sigma^2$ which can be positive or negative, however each of these solution branches are disconnected.
%**************************
\section{Thermodynamics}
%**************************

\end{document}
